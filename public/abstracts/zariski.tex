The WIYN 3.5m Telescope at Kitt Peak National Observatory hosts a
suite of optical and near-infrared instruments, including an extreme
precision, optical spectrograph, NEID, built for exoplanet radial
velocity studies. In order to achieve sub ms−1 precision, NEID has
strict requirements on survey efficiency, stellar image positioning,
and guiding performance, which have exceeded the native capabilities
of the telescope’s original pointing and tracking system. In order to improve the operational efficiency of the telescope we have developed a novel telescope pointing system, built on both a recurrent neural network and gradient-boosted tree, that does not rely on the usual pointing models (TPoint or other quasi-physical bases). We discuss the development of this system, how the intrinsic properties of the pointing problem inform our network design, and show preliminary results from our best models. We also discuss plans for the generalization of this framework, so that it can be applied at other sites.