Inflammatory bowel disease (IBD) is a chronic disease that increases the risk of colorectal cancer (CRC). Perturbations to sphingolipid metabolism are implicated in IBD and CRC. Acid ceramidase (AC) exhibits a role in promoting IBD and CRC by degrading ceramide for the generation of the pro-inflammatory bioactive lipid sphingosine-1-phosphate (S1P). Our lab has shown that loss of AC in myeloid cells (ACMYE) reduced inflammation in mice with acute DSS-induced colitis. We then investigated loss of AC in IL10 deficient mice, a chronic colitis model which recapitulates human disease. ACMYEIL10-/- and ACfl/flIL10-/- mice were generated and collected at 8 and 24 weeks of age to assess early and late stage disease, respectively. ACMYEIL10-/- mice were
protected from hallmarks of colitis and exhibited reduced expression (mRNA and protein) of inflammatory markers 24 weeks. Ceramides and sphingomyelins (SMs) were elevated in ACMYEIL10-/- mice, whereas lyso-SM, sphingoid bases, dhS1P, and S1P were reduced. FACS analysis revealed reduced infiltration of neutrophils and TypeM1 macrophages into colons of ACMYEIL10-/- mice. Interestingly, Th1 and Th17 T cell populations were reduced in ACMYEIL10-/- mice across various tissues. Cultured ACMYE bone marrow derived macrophages exhibited a failure to mount an S1P response against LPS stimulation and reduced secretion of chemokines and cytokines. Collectively these data suggest that loss of AC in myeloid cells protects from chronic inflammation and indicate a role for AC in mediating cell-to-cell crosstalk. Our findings highlight that AC may be a promising therapeutic target for IBD and reducing the risk of CRC. This work is supported by the National Institute of Diabetes and Digestive and Kidney Diseases R01 DK132079 (AJS).